\section{Conclusion}
We have tested our solution on a varied set of examples where we achieved good results! We did not yet deploy the compiler plugin for a large scale project, neither did we do a field study where we could have asked a group of developers to use the tool for some time. The limited testing is mostly due to the fact the described plugin has been realized as a small semester project with very strict timing constraints.

Even though fully dynamic languages have some other features like duck typing and monkey patching, deferring type errors are also a very important feature which we considered important enough to be implemented on its own. Some other implementations to make Scala more dynamic have already been made, as described in the related work section, but most of those approaches require additional runtime support which increases the overhead and potentially makes compiling and running the programs slower. The solution we present also shows that some dynamic features can be realized fairly easily with lightweight implementations which might be interesting for further development.

The implementation of the ScalaDyno compiler is open source and publicly available on github (https://github.com/scaladyno) as well as related information such as the documentation, a setup guide and a repository containing a set of examples.