\section{Related work}

\topic{Always-available static and dynamic feedback (DuctileJ)}

To add dynamic behaviour to the Java programming language, DuctileJ does a detyping transformation before the real typing phase. This detyping consists of converting the types of all the variables and fields, as well as all the method parameters, to {\ttfamily Object}, which is a the Java super class of all other types. In addition to those transformations, an runtime library {\ttfamily RT} is needed to support the following actions:
{\ttfamily
RT.newInstance("ClassName")\\
RT.select(instance, "fieldName")\\
RT.invoke("methodName", instanceName, args)\\
RT.assign(instance, "fieldName", value)\\
RT.cast("ClassName", instance)\\	
}
which are needed for the late runtime binding.
Such a transformation however requires the duplication of the typing phase in the compiler, the first to detype the code and collect possible error messages the second to do the normal typing. Also due to method overloading, method signatures needs to be mangled which means that that each original parameter need to be duplicated, one is needed to carry the type, the other to carry the value. Of course this creates some additional problems when working with pre-compiled libraries. However this transformation allows duck-typing which is also considered an important feature of dynamically typed languages.
\\
\\
\topic{DuctileScala: combined static and dynamic feedback for Scala}

Very similar to DuctileJ, DuctileScala also introduces a set of new Compiler phases ({\ttfamily signature, earlynamer, earlypackageobjects, earlytyper, detyper}) which perform detyping as well as other necessary transformation (e.g. signature mangling). Some transformations however are even more complicated due to implicit conversions (views) as well as pattern-matching which are unique features of Scala over Java.
\\
\\
\topic{Scala.js: Type-Directed Interoperability with Dynamically Typed Languages}

Instead of making the entire Scala language more dynamic, this approach focusses on integrating combined development with Scala an Javascript. A specific set of Scala classes is created which are only facade types for a corresponding Javascript object, those constructs can then either be compiled to dynamic Javascript or to Scala code. Some complications encountered were due to implicit conversions and the connections with existing Javascript libraries.
\\
\\
\topic{Scala-Virtualized: better support for embedded DSLs}

In analogy to hardware virtualization for virtual machines. The main idea of this approach is to be able to customize some build-in language features using virtual methods such as:\\*
{\ttfamily def \_\_ifThenElse[T](cond) }\\*
Similar to multi-stage programming Scala-Virtualized can be split in two parts, first the shallow embedding does type level computation using implicits followed by a deep embedding which enables to redefine build-in language constructs. "Lightweight Modular Staging (LMS) is a set of techniques and a core compiler frame-work for building embedded DSLs in Scala-Virtualized."
\\
\\
\topic{Javascript as an Embedded DSL}

Using the Lightweight Modular Staging technique described earlier, Javascript can be embedded as a DSL in Scala. Gradual typing allows granularity such that external JavaScript libraries don't need to be typed as they are only needed in later phases. In addition the DSL code can either be compiled to JavaScript or be used as Scala which means  that computation can either be done on the server side or the client side. 
\\
\\
\topic{(Equality proofs and deferred type errors)}

tbd needed?
\\
\\
\topic{(Closed Type families with overlapping equations)}

tbd needed?
\\
\newcommand{\DS}{\begin{sideways}DuctileScala\end{sideways}}
\newcommand{\hask}{\begin{sideways}Haskell\end{sideways}}
\newcommand{\dyn}{\begin{sideways}Dynamic\end{sideways}}
\newcommand{\SV}{\begin{sideways}Scala-Virtualized\end{sideways}}
\newcommand{\dart}{\begin{sideways}Dart\end{sideways}}
\newcommand{\DL}{\begin{sideways}dynamic languages\end{sideways}}
\newcommand{\SJS}{\begin{sideways}Scala-JS interoperatability\end{sideways}}
\newcommand{\JSLMS}{\begin{sideways}JS on LMS\end{sideways}}
\newcommand{\SD}{\begin{sideways}ScalaDyno\end{sideways}}

\begin{table*}[b]
\begin{tabular}{|l|l|l|l|l|l|l|l|l|l|}
	\hline
													& \DS    & \hask  & \dyn   & \SV    & \dart  & \DL    & \SJS   & \JSLMS & \SD \\
	\hline
    derferred class name resolution errors 			& \xmark & \xmark & \xmark & \xmark & \xmark & \cmark & \xmark & \xmark & \cmark \\
    \hline
    deferred method and field resolution errors 	& \cmark & \xmark & \cmark & \xmark & \xmark & \cmark & \cmark & \cmark & \cmark \\
    \hline
    deferred type resolution errors 				& \cmark & \cmark & \cmark & \cmark & \cmark & \cmark & \cmark & \cmark & \cmark \\
    \hline
    fast runtime 									& \xmark & \xmark & \xmark & \cmark & \cmark & \cmark & \xmark & \xmark & \cmark \\
    \hline
    work at compile time 							& \xmark & \cmark & \xmark & \cmark & \xmark & \xmark & \cmark & \cmark & \cmark \\
    \hline
\end{tabular}
\end{table*}