\section{Implementation}

A Scala compiler plugin is nothing more than an added compiler phase which the
as input the AST from the previous phase, transforms it corresponding to the
implemented behaviour by recursively running through, and possibly changing, all
AST nodes, and passes it on to the next phase.
In the normal static typer, typing error which happen at some branch in the tree
propagate  through the tree to the root nodes which mean that all parent nodes
are also erroneously typed and thus the full program tree will be erroneous and
no JVM code will be generated. To avoid this kind or error propagation we want
to intercept the erroneously typed branches and "cut them off"; prune them at a
level which is not harmful to the code generation such that the program can
still be compiled even if it contains name or type errors. Finally those pruned
branches will be replaced by trees representing Exceptions of type {\ttfamily
ErrorType} which will be thrown at runtime.
To achieve  this kind of behavior we first need to compiler-build-in reporter to
issue errors because they will break compilation right away. To do so we simply
change the reporter through reflection and and issue warnings instead of errors.
This conversion is however only done for naming and typing errors and not for
errors from other phases, e.g. parse errors. While converting those errors to
warnings already provides some static feedback to the programmer at compile time
it is also very useful to restate the error message whenever the program falls
in an erroneously branch during runtime. The easiest way to do this is to keep a
global mapping from positions to error messages which is later used to be
inserted into the AST as an Exception node message. Those exception messages, in
addition to the compiler warning output, will also be very useful for debugging.
As our plugin has the ability to change the tree we extends the class {\ttfamily
Transformer} form the New Scala Compiler tools which defines and provides the
method {\ttfamilydef transform(tree: Tree): Tree} which can be used as a default
traverser for the tree nodes which do not need a special treatment.

It is also very important to mention exceptions cannot be thrown at any place in
the code. (TODO... scope)
