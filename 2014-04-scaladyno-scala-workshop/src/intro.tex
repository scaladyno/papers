\section{Introduction}

In a statically typed programming language the type checker which runs at
compile time gives static feedback which provides some correctness guarantees of
its output, the compiled program. This method help to avoid some errors which
might only become visible at runtime for dynamically typed languages.Knowing the
object and class structures at compile time also help the compiler to do a lot
of optimizations such as object layout in the system memory.
On the other hand the restrictive type system might get in the way during agile
development where even small code changes would require a lot or effort to adapt
the surrounding code which depends on the modified code.

In this cases dynamically typed languages come in handy as they allow fast
prototyping because they have less strict typing rules. However without the
missing feedback no static guarantees are available if a program is correctly
typed or not. Those limits also create related problems such that refractoring
tools cannot detect and modify all the related code such that some if this work
needs to be done by hand. The runtime performance of dynamic languages is also
generally slower or at least cannot be optimized beyond a specific threshold due
to needed runtime type checks and monkey patching, where any field or method can
be added during the execution of a program.

In the case of an ideal programming language, static feedback should be
available but also be optional such that the programmer can decide whenever to
use a more dynamic or static language depending on the development phase his
work is currently in. For example in early development the programmer might want
more dynamic language features for fast prototyping and worry about
type-correctness everywhere only later. However the behaviour of programs or
parts of programs which are semantically equivalent to a type-correct program
should not be compiled to something different tha the original code and the
should have no or minimal runtime overhead.

There are two main approaches to address the problem of combining static and
dynamic language features. Either you start with a dynamic language and add
static features or you start with a static language and make it more dynamic.
The former represents a quite big change in the existing compiler as you would
need to create a new typer which would analyse the program at compile-time an
give some information about the typing. In general this is very hard as most
dynamically typed languages have very specific typing behaviour such as
duck-typing and monkey-patching which would make type analyzing even more
difficult than in some existing static typers. The latter might be a better
approach as you can reuse the existing static typer and only change behaviour
specific to the kind of error that you want to disregard.



