\documentclass[10pt, preprint]{sigplanconf}

\usepackage{amsmath}
\usepackage{url}
\usepackage[usenames,dvipsnames]{color}
\usepackage[english]{babel}
\usepackage{graphicx}
\usepackage{listings}
\usepackage[compact]{titlesec}
\usepackage{multirow}
\usepackage{color, colortbl}
\usepackage{fancyvrb}
\usepackage{shortvrb}
\usepackage{setspace}
\usepackage{flushend}
\usepackage{stfloats}
\usepackage{listings,xcolor,beramono}
\usepackage{tikz}
\usepackage{calc}
\usepackage[utf8x]{inputenc}
%% typesetting type system rules:
\usepackage{latexsym}
\usepackage{bcprules}
\usepackage[T1]{fontenc}
\usepackage{rotating}
\usepackage{amssymb}% http://ctan.org/pkg/amssymb
\usepackage{pifont}% http://ctan.org/pkg/pifont
 
% checkmark and cross
\newcommand{\cmark}{\ding{51}}%
\newcommand{\xmark}{\ding{55}}%

% Short verbatim
\DefineShortVerb{\|}
\fvset{fontsize=\small}

\definecolor{Gray}{gray}{0.9}
\definecolor{LightGray}{gray}{0.4}
\newcolumntype{g}{>{\columncolor{Gray}}r}

\newcommand\Small{\fontsize{8.5}{8.5}\selectfont}
\newcommand*\LSTfont{\Small\ttfamily\lsstyle}

% Divisible by
\DeclareRobustCommand{\divby}{%
  \mathrel{\vbox{\baselineskip.65ex\lineskiplimit0pt\hbox{.}\hbox{.}\hbox{.}}}%
}

\makeatletter
\newenvironment{btHighlight}[1][]
{\begingroup\tikzset{bt@Highlight@par/.style={#1}}\begin{lrbox}{\@tempboxa}}
{\end{lrbox}\bt@HL@box[bt@Highlight@par]{\@tempboxa}\endgroup}

\newcommand\btHL[1][]{%
  \begin{btHighlight}[#1]\bgroup\aftergroup\bt@HL@endenv%
}
\def\bt@HL@endenv{%
  %\end{btHighlight}
  \egroup
}
\newcommand{\bt@HL@box}[2][]{%
  \tikz[#1]{%
    \pgfpathrectangle{\pgfpoint{1pt}{0pt}}{\pgfpoint{\wd #2}{\ht #2}}%
    \pgfusepath{use as bounding box}%
    \node[anchor=base west, fill=black!10,outer sep=0pt,inner xsep=1pt, inner ysep=0pt, rounded corners=1pt, minimum height=\ht\strutbox+1pt,#1]{\raisebox{1pt}{\strut}\strut\usebox{#2}};
  }%
}

\g@addto@macro\normalsize{%
  \setlength\abovedisplayskip{-0.5em}
  \setlength\belowdisplayskip{1em}
  \setlength\abovedisplayshortskip{0.25em}
  \setlength\belowdisplayshortskip{0.25em}
}
\makeatother

% "define" Scala
\lstdefinelanguage{scala}{
  morekeywords={abstract,case,catch,class,def,%
    do,else,extends,false,final,finally,%
    for,if,implicit,import,match,mixin,%
    new,null,object,override,package,%
    private,protected,requires,return,sealed,%
    super,this,throw,trait,true,try,%
    type,val,var,while,with,yield},
  otherkeywords={=>,<-,<\%,<:,>:,\#,@},
  sensitive=true,
  morecomment=[l]{//},
  morecomment=[n]{/*}{*/},
  morestring=[b]",
  morestring=[b]',
  morestring=[b]""",
  moredelim=**[is][\btHL]{`}{`},
}

% Default settings for code listings
%\lstset{language=scala,showstringspaces=false,columns=flexible, basicstyle=\footnotesize \ttfamily}
\definecolor{ggray}{gray}{0.5}
\renewcommand{\ttdefault}{pcr}
\lstset{frame=tb,
  language=scala,
  aboveskip=3mm,
  belowskip=3mm,
  showstringspaces=false,
  columns=flexible,
  basicstyle={\footnotesize \ttfamily},
  %basicstyle=\LSTfont,
%% numbers:
%  numbers=none,
  numbers=left,
  %xleftmargin=2em,
  %framexleftmargin=1.5em,
%%% end numbers
  numberstyle=\tiny\color{gray},
  keywordstyle=\bfseries,
  commentstyle=\em\color{ggray},
  stringstyle=\em,
%  keywordstyle=\color{blue},
%  commentstyle=\color{OliveGreen},
%  stringstyle=\color{Purple},
  frame=single,
  breaklines=true,
  breakatwhitespace=true
  tabsize=3
}

%% Line numbers inside frame:
%% http://tex.stackexchange.com/questions/30504/how-can-i-properly-align-the-line-numbers-of-a-source-code-listing-with-the-marg
\makeatletter
\newlength{\linenumwidth} \setlength{\linenumwidth}{2em}% Redefine as required
\newlength{\numwidth}%
\setlength{\numwidth}{\widthof{\normalfont{\lst@numberstyle{9}}}}% Up to 2-digit (99) line numbers
\def\lst@PlaceNumber{%
  \makebox[\numwidth+0.3em][l]{%
    \makebox[\numwidth][r]{\normalfont\lst@numberstyle{\thelstnumber}}%
  }%
}
\makeatother


%%% How to prevent lstlisting from splitting code between pages?
%%% http://tex.stackexchange.com/questions/10141/how-to-prevent-lstlisting-from-splitting-code-between-pages
\lstnewenvironment{lstlisting-nobreak}[1][]%
{
   \noindent
   \vspace{-0.1em}
   \minipage{\linewidth}
   \vspace{0.12\baselineskip}
%    \lstset{basicstyle=\ttfamily\footnotesize,frame=single,#1}
}
{
   \vspace{-0.1em}
   \vspace{-0.18\baselineskip}
   \endminipage
}

% Footnote remember - http://anthony.liekens.net/index.php/LaTeX/MultipleFootnoteReferences
\newcommand{\footnoteremember}[2]{
\footnote{#2}
\newcounter{#1}
\setcounter{#1}{\value{footnote}}
}
\newcommand{\footnoterecall}[1]{
\footnotemark[\value{#1}]
}
% use:
% \footnoteremember{myfootnote}{This is my footnote}
% \footnoterecall{myfootnote}

% Packed item, so we don't waste space
\newenvironment{packed_item}{
\begin{itemize}
  \setlength{\itemsep}{1pt}
  \setlength{\parskip}{0.2pt}
  \setlength{\parsep}{0.2pt}
}{\end{itemize}}

% Packed enum, so we don't waste space
\newenvironment{packed_enum}{
\begin{enumerate}
  \setlength{\itemsep}{1pt}
  \setlength{\parskip}{0.2pt}
  \setlength{\parsep}{0.2pt}
}{\end{enumerate}}

% Review tools
%\newcommand{\topic}[1]{#1}
\newcommand{\topic}[1]{{\bf #1}}
% This marks comments
%\newcommand{\vu}[1]{{\color{BurntOrange}\framebox{{\bf VU}}\;#1}\;}
%\newcommand{\tr}[1]{{\color{Blue}{\bf \framebox{TR}\;#1}}\;}
%\newcommand{\as}[1]{{\color{Red}{\bf \framebox{AS}\;#1}}\;}
%\newcommand{\mo}[1]{{\color{OliveGreen}{\bf \framebox{MO}\;#1}}\;}
\newcommand{\textem}[1]{{\em#1}}
\newcommand{\newterm}[1]{{\em #1}}

% Paper HACKS
\setstretch{0.98}
\renewcommand{\bibfont}{\scriptsize}
%\renewcommand{\UrlFont}{\scriptsize}
\renewcommand*{\UrlFont}{\ttfamily\footnotesize\relax}

\begin{document}

\clubpenalty=10000
\widowpenalty = 10000

\titlebanner{DRAFT - Please do not circulate}        % These are ignored unless
\preprintfooter{DRAFT - Please do not circulate}     % 'preprint' option specified.

\title{ScalaDyno: Making Name Resolution and Type Checking Fault-Tolerant}

\authorinfo{Cédric Bastin \and Vlad Ureche \and Martin Odersky}
           {EPFL, Switzerland}
           {\{firstname.lastname\}@epfl.ch}

\maketitle

\section{Abstract}

%In the academic and the professional community it is agreed on the fact that both statically and dynamically typed languages have both benefits and drawbacks. Statically typed languages reject any code that they cannot prove to be correct. However this inflexible, especially when developers are prototyping incompatible changes to the code.
%On the other hand, dynamically typed languages give the developer more freedom by only performing the necessary checks at runtime, but give rise to code patterns that are impossible to check, such as monkey patching and using duck typing.

%It would be useful to have a programming language that combines the benefits of both static and dynamic typing. Such a language would allow agile development while also offering static guarantees for the final compilation.

%The ScalaDyno compiler plugin allows fast prototyping with the Scala programming language. This in done by deferring compilation errors to runtime and thus allowing byte code generation for partially incorrect programs. This reaps the benefits of static checking with the real-world requirements of fast prototyping and quick debugging.

%% TODO: Add Key insight + contribution
%\textbf{TODO: Add Key insight + contribution}

The ScalaDyno compiler plugin allows fast prototyping with the Scala programming language, in a way that combines the benefits of both statically and dynamically typed languages. Static name resolution and type checking prevent partially-correct code from being compiled and executed. Yet, allowing programmers to test critical paths in a program without worrying about the consistency of the entire code base is crucial to fast prototying and agile development. This is where ScalaDyno comes in: it allows partially-correct programs to be compiled and executed, while shifting compile-time errors to program runtime.

The key insight in ScalaDyno is that name and type errors affect limited areas of the code, which can be replaced by instructions reproducing the respective errors at runtime. This allows byte code generation and execution for partially correct programs, thus allowing Python or JavaScript-like fast prototyping in Scala. This is all done without sacrificing name resolution, full type checking and optimizations for the correct parts of the code -- only without them getting in the way of code changes. Finally, for release code or sensitive refactorings, runtime errors can be disabled, thus allowing full static name resolution and type checking typical of the Scala compiler.




%\category{E.2}{Object representation}{}

\keywords
Scala, dynamic typing, deferred type errors

\section{Introduction}

In a statically typed programming language the type checker which runs at
compile time gives static feedback which provides some correctness guarantees of
its output, the compiled program. This method help to avoid some errors which
might only become visible at runtime for dynamically typed languages.Knowing the
object and class structures at compile time also help the compiler to do a lot
of optimizations such as object layout in the system memory.
On the other hand the restrictive type system might get in the way during agile
development where even small code changes would require a lot or effort to adapt
the surrounding code which depends on the modified code.

In this cases dynamically typed languages come in handy as they allow fast
prototyping because they have less strict typing rules. However without the
missing feedback no static guarantees are available if a program is correctly
typed or not. Those limits also create related problems such that refractoring
tools cannot detect and modify all the related code such that some if this work
needs to be done by hand. The runtime performance of dynamic languages is also
generally slower or at least cannot be optimized beyond a specific threshold due
to needed runtime type checks and monkey patching, where any field or method can
be added during the execution of a program.

In the case of an ideal programming language, static feedback should be
available but also be optional such that the programmer can decide whenever to
use a more dynamic or static language depending on the development phase his
work is currently in. For example in early development the programmer might want
more dynamic language features for fast prototyping and worry about
type-correctness everywhere only later. However the behaviour of programs or
parts of programs which are semantically equivalent to a type-correct program
should not be compiled to something different tha the original code and the
should have no or minimal runtime overhead.

There are two main approaches to address the problem of combining static and
dynamic language features. Either you start with a dynamic language and add
static features or you start with a static language and make it more dynamic.
The former represents a quite big change in the existing compiler as you would
need to create a new typer which would analyse the program at compile-time an
give some information about the typing. In general this is very hard as most
dynamically typed languages have very specific typing behaviour such as
duck-typing and monkey-patching which would make type analyzing even more
difficult than in some existing static typers. The latter might be a better
approach as you can reuse the existing static typer and only change behaviour
specific to the kind of error that you want to disregard.





 \section{Motivation}

 Some effort has already been done to include dynamic behaviour into statically typed languages, however most of the approaches require big changes in the compiler and also sometimes runtime supporting libraries which slows down both compilation and execution due to the induced overhead. Our idea was to create a very lightweight tool which might not be as extensible and not cover as many cases as other alternatives but which still very fast, easy to implement and easy to use. Our choice for the Scala language was based on the fact that, even though this object-oriented, functional programming language has grown into a full real-world software development language, there is still missing an easy solution for fast development and testing. During the project we also wanted to explore other possible solution which such that the core compiler needed to be modified as least as possible. In the beginning we hoped that we could use the recently added trait {\ttfamily Dynamic} as well as in combination with implicit conversion and typed macros. Sadly we quickly needed to notify that implicit conversion don't apply for conversions to Dynamic. This means that implicit conversion won't automatically apply a conversion to a dynamic type whenever no statically correctly typed alternatives are available. Hence we needed to consider a more integrated solution such that we finally implemented a Scala compiler plugin which does the transformation directly on the AST. 

\section{Theory}

In Scala type {\ttfamily Nothing} is a subtype of anything and can thus substitute the return value of any type in a method return value for example.
Similarly the error type is the same than type {\ttfamily Nothing} with some bit-flags set hence it can basically replace any value, be it a field, return value or argument. This property is needed for our solution not to break the type checker.
Another important feature of bug reporting
Moreover do many typing errors depend on earlier errors in the code such as manipulations of an instance which could not be created in the first place. Those spurious error messages should not be considered as entirely different errors but should all relate to the first error. There are many advanced techniques nowadays to eliminate those "spurious" error messages. As cited in Ramsey's [reference] paper, error messages related to an erroneous identifier should be identified by having the corresponding identifier in the symbol type and by suppressing all following errors. This method helps to catch the errors, to issue only the most relevant error message and identify the root cause of a specific error. 

\section{Implementation}

A plugin for the Scala compiler is a separate program that can inject one or more compilation phases and alter the compiler options. In the case of ScalaDyno, we inject a single phase that takes as input the AST from the type checker phase (|typer|), traverses and cleans its erroneous statements by recursively running through all AST nodes. It also cleans up the symbol table by removing any erroneous symbol, namely any symbol whose type is either |ErrorType| or a derivate of it (e.g. |List[ErrorType]|). The final result is a pruned AST containing only references to correct symbols and symbol table which only contains correct symbols.

The normal behavior of the name resolution and type checking phases is to issue errors which prevent further compilation of the program. To achieve our goal of allowing partially correct programs to compile, we first need to prevent the compiler built-in |Reporter| from issuing errors which makes further compilation impossible. This can be done by changing the error reporter and transforming errors into warnings. This conversion is however only done for naming and typing errors and not for errors from other phases, e.g. parsing errors as well as overriding and abstract errors which are triggered by the refcheck phase and cannot be fixed by ScalaDyno. Since errors are converted to warnings, the programmer already receives some feedback during compilation, in the form of warnings. During reporting, we also record the suppressed errors, which we use to later patch the tree.

\begin{figure}[h]
\includegraphics[width=0.9\columnwidth]{compiler_structure.png}
\caption[structure sketch]
   {sketch of the comiler plugin and related behaviour}
\vspace{-15mm}
\end{figure}

In the type checker, typing errors which happen on some branches in the AST propagate outwards until a stable boundary is reached. Examples of stable boundaries are: the next statement in a block or the next definition in a class, trait, object or package. In order to clean up the tree, we remove the erroneous statements. Yet, as discussed before, we cannot allow the code to execute past an erroneous statement. To implement this, we actually replace erroneous statements by statements which throw exceptions. The message in the exception is the actual error output by the compiler for that particular part of the tree. This is implemented by matching source positions in the tree with source positions of the error messages. Positions are a mechanism by which the compiler records the position of each AST node in the source code. Errors also have positions attached, allowing their messages to point to the exact lines in the source code that triggered them. Therefore, based on the recorded messages and positions and the tree positions we can safely replace the tree nodes by exception-throwing statements.

There are a number of places where simply replacing an erroneous node by a statement doesn't work. Such cases are pattern matches, definitions inside classes, type-defining nodes and annotations. For these cases, we either have special rules which bubble up the statement (in the case of pattern matches) or we issue an error message that we can't properly clean up the tree and abort the compilation. While these errors could be mitigated, the additional complexity significantly burdens the plugin and does not bring significant benefit. Therefore we chose to focus on the most common errors which can easily be cleaned up.

\begin{figure}[h]
\begin{lstlisting-nobreak}
object Test {
   	def main(args: Array[String]) {
   		val c = Class1(3)
	   	val ret = c match {
	   		case Class1(1) => "one"
	   		case NoSuchClass(2) => "two"
	   		case Class1(3) => "three"
	   	}
	    println(ret)
	}
}
\end{lstlisting-nobreak}
\caption[pattern matching]
   {An example of pattern matching errors}
\end{figure}

The above code will result in a cleaned-up AST, after the work of the compiler plugin, with the node:

\begin{lstlisting-nobreak}
val ret: String = 'package'.this.error("
examples/Test3compilesMatch.scala:10: not found: value NoSuchClass
	   		case NoSuchClass(2) => "two"
                             ^

");
\end{lstlisting-nobreak}

 
\section{Related work}

\topic{Always-available static and dynamic feedback (DuctileJ)}

To add dynamic behaviour to the Java programming language, DuctileJ does a detyping transformation before the real typing phase. This detyping consists of converting the types of all the variables and fields, as well as all the method parameters, to {\ttfamily Object}, which is a the Java super class of all other types. In addition to those transformations, an runtime library {\ttfamily RT} is needed to support the following actions:
{\ttfamily
RT.newInstance("ClassName")\\
RT.select(instance, "fieldName")\\
RT.invoke("methodName", instanceName, args)\\
RT.assign(instance, "fieldName", value)\\
RT.cast("ClassName", instance)\\	
}
which are needed for the late runtime binding.
Such a transformation however requires the duplication of the typing phase in the compiler, the first to detype the code and collect possible error messages the second to do the normal typing. Also due to method overloading, method signatures needs to be mangled which means that that each original parameter need to be duplicated, one is needed to carry the type, the other to carry the value. Of course this creates some additional problems when working with pre-compiled libraries. However this transformation allows duck-typing which is also considered an important feature of dynamically typed languages.
\\
\\
\topic{DuctileScala: combined static and dynamic feedback for Scala}

Very similar to DuctileJ, DuctileScala also introduces a set of new Compiler phases ({\ttfamily signature, earlynamer, earlypackageobjects, earlytyper, detyper}) which perform detyping as well as other necessary transformation (e.g. signature mangling). Some transformations however are even more complicated due to implicit conversions (views) as well as pattern-matching which are unique features of Scala over Java.
\\
\\
\topic{Scala.js: Type-Directed Interoperability with Dynamically Typed Languages}

Instead of making the entire Scala language more dynamic, this approach focusses on integrating combined development with Scala an Javascript. A specific set of Scala classes is created which are only facade types for a corresponding Javascript object, those constructs can then either be compiled to dynamic Javascript or to Scala code. Some complications encountered were due to implicit conversions and the connections with existing Javascript libraries.
\\
\\
\topic{Scala-Virtualized: better support for embedded DSLs (domain specific languages)}

In analogy to hardware virtualization for virtual machines. The main idea of this approach is to be able to customize some build-in language features using virtual methods such as:\\*
{\ttfamily def \_\_ifThenElse[T](cond) }\\*
Similar to multi-stage programming Scala-Virtualized can be split in two parts, first the shallow embedding does type level computation using implicits followed by a deep embedding which enables to redefine build-in language constructs. "Lightweight Modular Staging (LMS) is a set of techniques and a core compiler frame-work for building embedded DSLs in Scala-Virtualized."
\\
\\
\topic{Javascript as an Embedded DSL}

Using the Lightweight Modular Staging technique described earlier, Javascript can be embedded as a DSL in Scala. Gradual typing allows granularity such that external JavaScript libraries do not need to be typed as they are only needed in later phases. In addition the DSL code can either be compiled to JavaScript or be used as Scala which means  that computation can either be done on the server side or the client side. 
\\
\\
\topic{(Equality proofs and deferred type errors)}

tbd needed?
\\
\\
\topic{(Closed Type families with overlapping equations)}

tbd needed?
\\
\newcommand{\DS}{\begin{sideways}DuctileScala\end{sideways}}
\newcommand{\hask}{\begin{sideways}Haskell\end{sideways}}
\newcommand{\dyn}{\begin{sideways}Dynamic trait\end{sideways}}
\newcommand{\SV}{\begin{sideways}Scala-Virtualized\end{sideways}}
\newcommand{\dart}{\begin{sideways}Dart\end{sideways}}
\newcommand{\DL}{\begin{sideways}dynamic languages\end{sideways}}
\newcommand{\SJS}{\begin{sideways}Scala-JS interoperatability\end{sideways}}
\newcommand{\JSLMS}{\begin{sideways}JS on LMS\end{sideways}}
\newcommand{\SD}{\begin{sideways}ScalaDyno\end{sideways}}

\begin{table*}[b]
\begin{tabular}{|l|l|l|l|l|l|l|l|l|l|}
	\hline
													& \DS    & \hask  & \dyn   & \SV    & \dart  & \DL    & \SJS   & \JSLMS & \SD \\
	\hline
    derferred class name resolution errors 			& \xmark & \xmark & \xmark & \xmark & \xmark & \cmark & \xmark & \xmark & \cmark \\
    \hline
    deferred method and field resolution errors 	& \cmark & \xmark & \cmark & \xmark & \xmark & \cmark & \cmark & \cmark & \cmark \\
    \hline
    deferred type resolution errors 				& \cmark & \cmark & \cmark & \cmark & \cmark & \cmark & \cmark & \cmark & \cmark \\
    \hline
    fast runtime 									& \xmark & \xmark & \xmark & \cmark & \cmark & \cmark & \xmark & \xmark & \cmark \\
    \hline
    work at compile time 							& \xmark & \cmark & \xmark & \cmark & \xmark & \xmark & \cmark & \cmark & \cmark \\
    \hline
\end{tabular}
\end{table*}

\section{Conclusion}
We have tested our solution on a varied set of examples where we achieved good results! We did not yet deploy the compiler plugin for a large scale project, neither did we do a field study where we could have asked a group of developers to use the tool for some time. The limited testing is mostly due to the fact the described plugin has been realized as a small semester project with very strict timing constraints.

Even though fully dynamic languages have some other features like duck typing and monkey patching, deferring type errors are also a very important feature which we considered important enough to be implemented on its own. Some other implementations to make Scala more dynamic have already been made, as described in the related work section, but most of those approaches require additional runtime support which increases the overhead and potentially makes compiling and running the programs slower. The solution we present also shows that some dynamic features can be realized fairly easily with lightweight implementations which might be interesting for further development.

The implementation of the ScalaDyno compiler is open source and publicly available on github (https://github.com/scaladyno) as well as related information such as the documentation, a setup guide and a repository containing a set of examples.

 \section{Future Work}

The described method for deferring type errors could be extended or combined with other dynamic features such as duck typing. In extension to the existing related work this might be possible by creating {\ttfamily Dynamic} wrapper classes for every class and detyping all instances, fields and method arguments to {\ttfamily Dynamic} such that field resolution could be done at runtime. Of course this would require additional work such as method signature mangling as well as finding an adequate solution for implicits.
\\
One additional feature which is not included in ScalaDyno is the possibility to be less strict about class inheritance such that abstract methods would not need to be implemented by sub-classes unless they were needed at runtime. This would need to be handled in during the {\ttfamily refcheck} phase of the Scala compiler.

 \section{Acknowledgments}

% Vlad: I am an author, which is more than a acknowledgement
%  At this point I wanted to thank Vlad Ureche, my supervisor, for his great help with this semester project. By taking time off in his busy schedule, we were able to have weekly meetings where he gave me very valuable feedback without which this project would not have been successful.

The authors would like to thank Damien Engels for his help with setting up the infrastructure, the reviewers who read the initial draft of the paper and gave us useful feedback and the members of the Programming Languages Laboratory in EPFL with whom we had very frutiful discussions on the fundamentals of avoiding type checking errors (and especially to Hubert Plociniczak, Eugene Burmako and Sandro Stucki). \\

\bibliographystyle{abbrvnat}
\bibliography{main} 

\end{document}
