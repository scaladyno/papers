\documentclass[10pt, preprint]{sigplanconf}

\usepackage{amsmath}
\usepackage{url}
\usepackage[usenames,dvipsnames]{color}
\usepackage[english]{babel}
\usepackage{graphicx}
\usepackage{listings}
\usepackage[compact]{titlesec}
\usepackage{multirow}
\usepackage{color, colortbl}
\usepackage{fancyvrb}
\usepackage{shortvrb}
\usepackage{setspace}
\usepackage{flushend}
\usepackage{stfloats}
\usepackage{listings,xcolor,beramono}
\usepackage{tikz}
\usepackage{calc}
\usepackage[utf8x]{inputenc}
%% typesetting type system rules:
\usepackage{latexsym}
\usepackage{bcprules}
\usepackage[T1]{fontenc}
\usepackage{rotating}
\usepackage{amssymb}% http://ctan.org/pkg/amssymb
\usepackage{pifont}% http://ctan.org/pkg/pifont
 
% checkmark and cross
\newcommand{\cmark}{\ding{51}}%
\newcommand{\xmark}{\ding{55}}%

% Short verbatim
\DefineShortVerb{\|}
\fvset{fontsize=\small}

\definecolor{Gray}{gray}{0.9}
\definecolor{LightGray}{gray}{0.4}
\newcolumntype{g}{>{\columncolor{Gray}}r}

\newcommand\Small{\fontsize{8.5}{8.5}\selectfont}
\newcommand*\LSTfont{\Small\ttfamily\lsstyle}

% Divisible by
\DeclareRobustCommand{\divby}{%
  \mathrel{\vbox{\baselineskip.65ex\lineskiplimit0pt\hbox{.}\hbox{.}\hbox{.}}}%
}

\makeatletter
\newenvironment{btHighlight}[1][]
{\begingroup\tikzset{bt@Highlight@par/.style={#1}}\begin{lrbox}{\@tempboxa}}
{\end{lrbox}\bt@HL@box[bt@Highlight@par]{\@tempboxa}\endgroup}

\newcommand\btHL[1][]{%
  \begin{btHighlight}[#1]\bgroup\aftergroup\bt@HL@endenv%
}
\def\bt@HL@endenv{%
  %\end{btHighlight}
  \egroup
}
\newcommand{\bt@HL@box}[2][]{%
  \tikz[#1]{%
    \pgfpathrectangle{\pgfpoint{1pt}{0pt}}{\pgfpoint{\wd #2}{\ht #2}}%
    \pgfusepath{use as bounding box}%
    \node[anchor=base west, fill=black!10,outer sep=0pt,inner xsep=1pt, inner ysep=0pt, rounded corners=1pt, minimum height=\ht\strutbox+1pt,#1]{\raisebox{1pt}{\strut}\strut\usebox{#2}};
  }%
}

\g@addto@macro\normalsize{%
  \setlength\abovedisplayskip{-0.5em}
  \setlength\belowdisplayskip{1em}
  \setlength\abovedisplayshortskip{0.25em}
  \setlength\belowdisplayshortskip{0.25em}
}
\makeatother

% "define" Scala
\lstdefinelanguage{scala}{
  morekeywords={abstract,case,catch,class,def,%
    do,else,extends,false,final,finally,%
    for,if,implicit,import,match,mixin,%
    new,null,object,override,package,%
    private,protected,requires,return,sealed,%
    super,this,throw,trait,true,try,%
    type,val,var,while,with,yield},
  otherkeywords={=>,<-,<\%,<:,>:,\#,@},
  sensitive=true,
  morecomment=[l]{//},
  morecomment=[n]{/*}{*/},
  morestring=[b]",
  morestring=[b]',
  morestring=[b]""",
  moredelim=**[is][\btHL]{`}{`},
}

% Default settings for code listings
%\lstset{language=scala,showstringspaces=false,columns=flexible, basicstyle=\footnotesize \ttfamily}
\definecolor{ggray}{gray}{0.5}
\renewcommand{\ttdefault}{pcr}
\lstset{frame=tb,
  language=scala,
  aboveskip=3mm,
  belowskip=3mm,
  showstringspaces=false,
  columns=flexible,
  basicstyle={\footnotesize \ttfamily},
  %basicstyle=\LSTfont,
%% numbers:
%  numbers=none,
  numbers=left,
  %xleftmargin=2em,
  %framexleftmargin=1.5em,
%%% end numbers
  numberstyle=\tiny\color{gray},
  keywordstyle=\bfseries,
  commentstyle=\em\color{ggray},
  stringstyle=\em,
%  keywordstyle=\color{blue},
%  commentstyle=\color{OliveGreen},
%  stringstyle=\color{Purple},
  frame=single,
  breaklines=true,
  breakatwhitespace=true
  tabsize=3
}

%% Line numbers inside frame:
%% http://tex.stackexchange.com/questions/30504/how-can-i-properly-align-the-line-numbers-of-a-source-code-listing-with-the-marg
\makeatletter
\newlength{\linenumwidth} \setlength{\linenumwidth}{2em}% Redefine as required
\newlength{\numwidth}%
\setlength{\numwidth}{\widthof{\normalfont{\lst@numberstyle{9}}}}% Up to 2-digit (99) line numbers
\def\lst@PlaceNumber{%
  \makebox[\numwidth+0.3em][l]{%
    \makebox[\numwidth][r]{\normalfont\lst@numberstyle{\thelstnumber}}%
  }%
}
\makeatother


%%% How to prevent lstlisting from splitting code between pages?
%%% http://tex.stackexchange.com/questions/10141/how-to-prevent-lstlisting-from-splitting-code-between-pages
\lstnewenvironment{lstlisting-nobreak}[1][]%
{
   \noindent
   \vspace{-0.1em}
   \minipage{\linewidth}
   \vspace{0.12\baselineskip}
%    \lstset{basicstyle=\ttfamily\footnotesize,frame=single,#1}
}
{
   \vspace{-0.1em}
   \vspace{-0.18\baselineskip}
   \endminipage
}

% Footnote remember - http://anthony.liekens.net/index.php/LaTeX/MultipleFootnoteReferences
\newcommand{\footnoteremember}[2]{
\footnote{#2}
\newcounter{#1}
\setcounter{#1}{\value{footnote}}
}
\newcommand{\footnoterecall}[1]{
\footnotemark[\value{#1}]
}
% use:
% \footnoteremember{myfootnote}{This is my footnote}
% \footnoterecall{myfootnote}

% Packed item, so we don't waste space
\newenvironment{packed_item}{
\begin{itemize}
  \setlength{\itemsep}{1pt}
  \setlength{\parskip}{0.2pt}
  \setlength{\parsep}{0.2pt}
}{\end{itemize}}

% Packed enum, so we don't waste space
\newenvironment{packed_enum}{
\begin{enumerate}
  \setlength{\itemsep}{1pt}
  \setlength{\parskip}{0.2pt}
  \setlength{\parsep}{0.2pt}
}{\end{enumerate}}

% Review tools
%\newcommand{\topic}[1]{#1}
\newcommand{\topic}[1]{{\bf #1}}
% This marks comments
%\newcommand{\vu}[1]{{\color{BurntOrange}\framebox{{\bf VU}}\;#1}\;}
%\newcommand{\tr}[1]{{\color{Blue}{\bf \framebox{TR}\;#1}}\;}
%\newcommand{\as}[1]{{\color{Red}{\bf \framebox{AS}\;#1}}\;}
%\newcommand{\mo}[1]{{\color{OliveGreen}{\bf \framebox{MO}\;#1}}\;}
\newcommand{\textem}[1]{{\em#1}}
\newcommand{\newterm}[1]{{\em #1}}

% Paper HACKS
\setstretch{0.98}
\renewcommand{\bibfont}{\scriptsize}
%\renewcommand{\UrlFont}{\scriptsize}
\renewcommand*{\UrlFont}{\ttfamily\footnotesize\relax}

\begin{document}

\clubpenalty=10000
\widowpenalty = 10000

\titlebanner{DRAFT - Please do not circulate}        % These are ignored unless
\preprintfooter{DRAFT - Please do not circulate}     % 'preprint' option specified.

\title{ScalaDyno: Making Name Resolution and Type Checking Fault-Tolerant}

\authorinfo{Cédric Bastin \and Vlad Ureche \and Martin Odersky}
           {EPFL, Switzerland}
           {\{firstname.lastname\}@epfl.ch}

\maketitle

\section{Abstract}

%In the academic and the professional community it is agreed on the fact that both statically and dynamically typed languages have both benefits and drawbacks. Statically typed languages reject any code that they cannot prove to be correct. However this inflexible, especially when developers are prototyping incompatible changes to the code.
%On the other hand, dynamically typed languages give the developer more freedom by only performing the necessary checks at runtime, but give rise to code patterns that are impossible to check, such as monkey patching and using duck typing.

%It would be useful to have a programming language that combines the benefits of both static and dynamic typing. Such a language would allow agile development while also offering static guarantees for the final compilation.

%The ScalaDyno compiler plugin allows fast prototyping with the Scala programming language. This in done by deferring compilation errors to runtime and thus allowing byte code generation for partially incorrect programs. This reaps the benefits of static checking with the real-world requirements of fast prototyping and quick debugging.

%% TODO: Add Key insight + contribution
%\textbf{TODO: Add Key insight + contribution}

The ScalaDyno compiler plugin allows fast prototyping with the Scala programming language, in a way that combines the benefits of both statically and dynamically typed languages. Static name resolution and type checking prevent partially-correct code from being compiled and executed. Yet, allowing programmers to test critical paths in a program without worrying about the entire code base is crucial to fast prototying and agile development. This is where ScalaDyno comes in: it allows partially-correct programs to be compiled and executed, while shifting compile-time errors to program runtime.

The key insight in ScalaDyno is that name and type errors affect limited areas of the code, which can be replaced by instructions reproducing the respective errors at runtime. This allows byte code generation and execution for partially correct programs, thus allowing Python or JavaScript-like fast prototyping in Scala. This is all done without sacrificing name resolution, full type checking and optimizations for the correct parts of the code -- only without them getting in the way of code changes. Finally, for release code or sensitive refactorings, runtime errors can be disabled, thus allowing full static name resolution and type checking typical of the Scala compiler.




%\category{E.2}{Object representation}{}

\keywords
Scala, dynamic typing, deferred type errors

\section{Introduction}

\topic{Generics allow programmers} to describe algorithms and data structures irrespective of the data they operate on. This enables code reuse and type safety. For the programmer, generic code, which uses parametric polymorphism, exposes a uniform and type safe interface that can be reused in different contexts, while offering the same behavior and guarantees. This increases productivity and improves code quality. Modern programming languages offer generic collections, such as linked lists, array buffers or maps as part of their standard libraries.

\topic{But despite the uniformity exposed to programmers, the lower level translation of generic code struggles with fundamentally non-uniform data.} To illustrate the problem, we can analyze the |contains| method of a linked list parameterized on the element type, |T|, written in the Scala programming language:

\begin{lstlisting-nobreak}
 def contains(element: T): Boolean = ...
\end{lstlisting-nobreak}

When translating the |contains| method to lower level code, such as assembly or bytecode targeting a virtual machine, a compiler needs to know the exact type of the parameter, so it can be correctly retrieved from the stack, registers or read from memory. But since the list is generic, the type parameter |T| can have different bindings, depending on the context, ranging from a byte to a floating point number or a pointer to a heap object, each with different sizes and semantics. So the compiler needs to bridge the gap between the uniform interface and the non-uniform low level implementation.

\topic{Languages targeting the Java Virtual Machine \cite{x10-www, scala-www, kotlin-www, ceylon-www} use a technique called erasure \cite{java-erasure} to bridge the gap between the high level generics and low level bytecode.} Erasure of generics makes it possible to have a uniform translation at the low level by coercing the generic data to be passed by reference. This is naturally compatible with heap objects in the language. Yet, since primitive types such as integers and floating point numbers are not part of the class hierarchy, primitive values need to be contained in objects, so they can be passed to low level generic code. This coercing process is called boxing. For each primitive value, boxing creates a special object to store it. The opposite process, which extracts the primitive value is called unboxing. Despite simplifying the low level translation of generics, boxing and unboxing bring significant performance drawbacks: values are accessed indirectly from heap objects, redundant object headers are allocated along with values and since values are stored as objects, the cache locality is not guaranteed anymore. This is a major concern for performance-aware code, and was addressed in the Scala compiler using the specialization transformation.

\topic{Specialization \cite{specialization-iuli} is an optimized transformation in the Scala compiler meant to improve the performance of generic code.} In practice, specialization speeds up code execution by an order of magnitude \cite{erik-spire} by creating multiple transaltions of the generic code, adapted for each primitive type. This allows executing generic code without the need for boxing and unboxing operations. For a piece of generic code depending on a single type parameter, specialization will generate 10 variants of the code, 9 for the primitive types in Scala and one for objects. Unfortunately this doesn't scale well: for a class with 3 type parameters, such as |Tuple3| or |Function2|, specialization produces $10^3$ classes, which is too much bytecode to include in the standard library.

\topic{Miniboxing aims at reducing the amount of bytecode produced by specialization,} by trading off a small percent of the execution speed. With miniboxing, classes such as |Tuple3| and |Function2| are specialized with just 8 variants in the low level code. The key insight of the miniboxing translation is that all primitive types can be stored in a tagged-union-like structure \cite{tagged-unions-lua}, thus reducing the code duplication to 2 classes per type parameter. The miniboxing transformation has been shown to perform on par with specialization on microbenchmarks \cite{miniboxing}, but to date, it has not been explored for high level language constructions, such as closures, implicits and mix-ins.

Scala provides high-level generic collections, promoting type-safe code reuse and standardization. Yet, their performance degrades when they are used for primitive types, since collections are not specialized due to the bytecode bloat. This has led to many discussions on the mailing lists, with users being unhappy about the slowdowns incurred for primitive types. Therefore, miniboxing is a good candidate for specializing Scala collections, as it reaps the benefits of specialization without creating prohibitive amounts of bytcode, this making the library both fast and portable.

This paper explores the interaction of miniboxing with different high level language constructs and with patterns used in the Scala collections, such as closures, implicits, mix-ins and the builder pattern. In this context, it makes the following contributions:

\begin{packed_item}
\item explains the miniboxing transformation in the Scala compiler from a programmer's perspective;
\item shows how the miniboxing transformation interacts with the high-level constructs in the Scala language;
\item presents benchmarks that show the miniboxing plugin can produce optimal code for high-level constructs.
\end{packed_item}

The next section will describe the miniboxing transformation.

 \section{Approach}

%   A typical bug-fix scenario includes adding an extra argument in several methods. Doing so manually by modifying the signature can lead to inconsistencies in the code since multiple call sites would need to be updated. In a statically typed language it would be impossible to prototype such a change since it would be rejected by the type checker. In this case it would be desirable to be able to execute only the path in the program relevant to the bug, while ignoring overall consistency until the fix is working correctly.
%
%   When refactoring part of the code which consists for example in changing the signature of a method or a constructor all the corresponding calls which use this changed construct also need to be filled in with default arguments. This however mostly need to be done by hand with require a big effort from the programmer mostly because the given described process might need to be done several time during design iterations. Of course the given problem could be solved by overloading the given function definition but again this creates some unnecessary overhead due to the reason described above.
%
%   Secondly, during refactoring and release preparation, a more static approach is preferrable, to allow the developer to confidently make changes without the risk of breaking the code. For example when refacting part of the code which consists in changing the signature of a method or a constructor all the corresponding calls which use this changed construct also need to be filled in with default arguments. However this needs to be done mostly by hand, thus requiring a big effort from the programmer mostly because the described process might need to be done several times during design iterations. Of course the given problem could be solved by overloading the given function definition but again this creates some unnecessary overhead due to the reasons described above. Hence it would be very useful to be able to execute paths of the program which do not include any of those wrong signatures being called. %In the case where such branches would be traversed.

% \textbf{This is more like explaining how things happened, rather than describing the motivation.}

% Effort has been invested in including dynamic behaviour into statically typed languages such as, for example, in Haskell and Scala. However, most of the approaches require big changes in the compiler and also supporting runtime libraries which slows down both compilation and execution due to the induced overhead. Our idea was to create a very lightweight tool which might not be as extensible and not cover as many cases as other alternatives but which can still be very fast, easy to implement and easy to use. Our choice for the Scala language was based on the fact that, even though this object-oriented, functional programming language has grown into a full real-world software development language, there is still missing an easy solution for fast development and testing. During the project we also wanted to explore other possible solutions such that the core compiler would need the fewest changes. In the beginning we hoped that we could use the recently added trait {\ttfamily Dynamic} in combination with implicit conversion and typed macros. Sadly we quickly needed to notify that implicit conversion do not apply for conversions to Dynamic. This means that implicit conversion won't automatically apply a conversion to a dynamic type whenever no statically correctly typed alternatives are available. Hence we needed to consider a more integrated solution such that we finally implemented a Scala compiler plugin which does the transformation directly on the AST.

Our goal was to create a modified version of the Scala compiler which would allow quick prototyping such that agile development can be made which means that many code iteration do not require making the entire code conforming to the current changes.

Mostly those prototypes will include methods with missing or additional arguments or a change in their types. Another example would be adding or removing methods or fields from a class without changing the code of the instances everywhere.
In addition to those requirements our solution should not change the semantics of correct code for example should the use of implicits still be possible and a correct program should compile to the same code as with the standard compiler

When looking at existing approaches out there one can quickly distinguish few main different approaches. One could use a dynamic language and augment it with type annotations which could be use to give static feedback at compile time without changing their dynamic behaviour such that only warning are generated at compile time and no errors.

When moving all checks to runtime, expensive reflection calls are needed to analyze the class structure at runtime which creates a big overhead especially for large projects.

To do some resolution at compile time one could use intermediate techniques such as the use of the |Dynamic| trait which acts as a proxy such that the names which can be resolved at compile time are whereas the other ones are deferred to runtime.

In our approach we want to collect the error message during compilation. Because an erroneous tree cannot continue the compilation we need to clean up the tree and remove erroneous nodes after having collected their corresponding error messages. The suppressed error should however not be lost but used as important debug information at compile- and run-time. With this step done we should be able to continue compilation of the program to byte code.

\section{Theory}

%In the Scala language the the type {\ttfamily Nothing} is a subtype of all other classes and can thus substitute the return value of any type in a method return value for example.
%Similarly the subtyping relations for the error type are the same as for type {\ttfamily Nothing}, which means that they are equivalent except for some bit-flags. Hence an instance of error type, just as one of type {\ttfamily Nothing}, can replace any other value, be it a field, return value or argument without changing or breaking the typing rules.
%Moreover do many typing errors depend on earlier errors such as an object which could not be instantiated and its subsequent field and method accesses. Those errors should not be considered as different ones as they relate to another error. There are many advanced techniques nowadays to eliminate those "spurious" error messages. As cited in Ramsey's [reference] paper, error messages related to an erroneous identifier should be traceable by having their corresponding identifier in the symbol table with type error and by suppressing all the following errors which relate to it. This method helps to catch the spurious errors and to issue only the most relevant error message which identifies the root cause of a specific error.

In order to remove erroneous parts of the AST we have to make some main assumptions which define the theory behind those changes.
First of all one can note that type errors in Scala, including their side-effects, are localized; they are bound inside a scope defined by a block which might be a method or class body. However those localized errors can still trigger other (localized) errors in other parts of the code due to instantiations and method calls.

In the current compiler however one single compile-time error still halts the entire compilation process so there should be an easy way to get rid of this behaviour by replacing erroneous nodes by something else. Indeed the method gen.mkSysErrorCall(arg:String) allows to create runtime exceptions of type |Nothing|, which can hence be used to replace entire subtrees of the AST which are erroneous. In the Scala language, type |Nothing| represents the bottom type of type system which means that is it the subtype of any other class and can hence be used as return argument for any method without breaking the subtyping relations.

After this cleanup of the AST, which removed all erroneously typed nodes from the tree, following phases of the compiler are thus able to perform additional transformations on the AST which does not include any erroneous branches any more.

Even if some of the type errors have cascading behaviour there might still be parts in the code which could be executed correctly. We want to keep those parts for the code generating process which is done by pruning the nodes of the AST which are of ErrorType and replace them by a corresponding system error of type Nothing which contain the error message that is usually given at compile-time.
 

\section{Implementation}

A plugin for the Scala compiler is an added compiler phase which takes as input the AST from the previous phase, transforms it corresponding to the implemented behaviour by recursively running through, and possibly changing, all AST nodes, and passes it on to the next phase.

In the normal static typer, typing errors which happen on some branches in the tree propagate through the tree to the root nodes which implies that all parent nodes are also erroneously typed and thus the full program tree will be erroneous and no JVM code will be generated. To avoid this kind or error propagation we want to intercept the erroneously typed branches and "cut them off"; prune them at a level which is not harmful to the code generation such that the program can still be compiled even if it contains name or type errors. Finally those pruned
branches will be replaced by trees representing Exceptions of type \texttt{ErrorType} which will be thrown at runtime.

To achieve  this kind of behavior we first need to compiler-build-in reporter to issue errors because they will break compilation right away. To do so we simply change the reporter through reflection and issue warnings instead of errors. This conversion is however only done for naming and typing errors and not for errors from other phases, e.g. parse errors. While converting those errors to warnings already provides some static feedback to the programmer at compile time and it is also very useful to restate the error message whenever the program falls
in an erroneously branch during runtime. The easiest way to do this is to keep a global mapping from positions to error messages which is later used to be inserted into the AST as an Exception node message. Those exception messages, in addition to the compiler warning output, will also be very useful for debugging.

As our plugin has the ability to change the tree we extends the class \texttt{Transformer} form the New Scala Compiler tools which defines and provides the method \texttt{transform(tree: Tree): Tree} which can be used as a default traverser for the tree nodes which do not need a special treatment.

It is also very important to mention that exceptions cannot be thrown at any place in the code i.e. that AST subtrees can be replaced by system errors. For example does it not make sense to pass an error as a method argument nor to match on a pattern which is of error type. For this reason ScalaDyno propagates errors declaration to specific levels in the AST where they are safe to be thrown. Such code levels include \texttt{DefTree}s such as class definitions, member definition and value definitions as well as \texttt{pattern match} expressions. As discussed in {theory}, all errors related to the same initial error should not generate additional warnings and should reference the root error during runtime failures. This is also achieved implicitly by taking the errors messages generated by the Scala compiler and their corresponding unique positions, saving those in a mapping such that no duplicates can exist.

 
\section{Related work}

Several approaches to enabling faster prototyping are currently in use: (1) dynamic languages with checking, (2) reflection, (3) proxies and (4) moving type computations to runtime.

\subsection{Dynamic Languages with Checking}

A dynamic language can be augmented it with type annotations which can then be used to give static feedback at compile time. These annotations would be optional, and the checks would only trigger if both the actual and the expected type are annotated, as the Dart programming language does \cite{dart}. Yet, such approaches are still fundamentally dynamic, and thus allow patterns such as monkey patching and duck typing, which, once used, make the code base impossible to check completely.
Typed Racket \cite{racket} as well as Strongtalk (a Smalltalk dialect with optional static typing support) \cite{strongtalk}, a , are other examples of brining static typing to a dynamic language.

\subsection{Using Reflection}

A second approach is completely switching to the use of reflection, practically turning a statically typed language such as Java or Scala into a dynamic language. This has been implemented in DuctileJ and DuctileScala \cite{ductilej,ductilescala}. Yet this approach makes heavy use of reflection and is unable to resolve implicits. This makes it unsuitable for our use case, as it introduces significant overheads for correct programs and it potentially prevents correct programs using implicits from running at all.

To add dynamic behavior to the Java programming language, DuctileJ \cite{ductilej} does a detyping transformation before the real typing phase. This detyping consists of converting the types of all the variables and fields, as well as all the method parameters, to |Object|, which is a the Java super class of any other type. In addition to those transformations, a runtime library |RT| is needed to support late binding:

\begin{lstlisting-nobreak}
RT.newInstance("ClassName")
RT.select(instance, "fieldName")
RT.invoke("methodName", instanceName, args)
RT.assign(instance, "fieldName", value)
RT.cast("ClassName", instance)
\end{lstlisting-nobreak}

Such a transformation however requires the duplication of the typing phase in the compiler, the first to detype the code and collect possible error messages the second to do the standard typing on the modified tree. Also due to method overloading, method signatures needs to be mangled which means that each original parameter needs to be duplicated, one is needed to carry the type, the other to carry the value. Of course this creates some additional problems when working with pre-compiled libraries. However this transformation allows duck-typing which is also considered an important feature of dynamically typed languages.

Very similar to DuctileJ, DuctileScala \cite{ductilescala} also introduces a set of new compiler phases (|signature|, |earlynamer|, |earlypackageobjects|, |earlytyper|, |detyper|) which perform detyping as well as other necessary transformation (e.g. signature mangling). Some transformations however are even more complicated due to implicit conversions (views) as well as pattern-matching which are unique features of Scala compared to Java.

\subsection{Proxies}

A third approach is using an proxy technique, such as the |Dynamic| trait in Scala \cite{dynamic} which acts as a proxy that allows rewriting unresolved methods to more general proxies, which can later use reflection to call the correct methods. Such approaches have been used to allow interoperability between Scala and JavaScript \cite{scala-js}, but they are not able to handle all cases necessary for prototyping, notably they require correct name and type resolution to work properly. Unfortunately, in many practical scenarios, once an error has occurred, type inference doesn't kick in anymore and is not able to infer the |Dynamic| marker in a value's type, such that methods may be called on it.

Instead of making the entire Scala language more dynamic, the scala-js approach \cite{scala-js} focuses on integrating combined development with Scala an Javascript. A specific set of Scala classes is created which are only facade types for a corresponding Javascript object, those constructs can then either be compiled to dynamic Javascript or to Scala code. Some complications encountered were due to implicit conversions and the connections with existing Javascript libraries.

A somewhat similar approach is taken by scala-virtualized \cite{scala-virtualized}: using the analogy to hardware virtualization for program languages, one can customize the build-in language constructs by transforming them into method calls:

\begin{lstlisting-nobreak}
def __ifThenElse[T](cond: Rep[Boolean], thene: => Rep[T], elsee: => Rep[T])
\end{lstlisting-nobreak}

Scala-Virtualized enables advanced multi-stage programming in Scala, using the Lightweight Modular Staging (LMS) framework. The LMS framework has been very successful at optimizing embedded domain-specific languages (DSLs). One such DSL is JavaScript, which can be embedded in Scala \cite{greg-js-dsl}. Gradual typing allows granularity such that external JavaScript libraries do not need to be typed as they are only needed in later phases. In addition the DSL code can either be compiled to JavaScript or be used as Scala which means  that computation can either be done on the server side or the client side.

\subsection{Deferred Type Errors}

The GHC\cite{haskell-deferred-type-errors} compiler allows the addition of equality proofs to the system FC intermediate language. Equality proofs can be completely erased so that they induce no runtime overhead. They are also first class citizens such that proof-as-values allow to defer type errors to runtime so partially type incorrect programs can compile and execute.


\newcommand{\DS}{\begin{sideways}DuctileScala\end{sideways}}
\newcommand{\hask}{\begin{sideways}Haskell\end{sideways}}
\newcommand{\dyn}{\begin{sideways}Dynamic / Scala-JS\;\end{sideways}}
\newcommand{\SV}{\begin{sideways}Scala-Virtualized\end{sideways}}
\newcommand{\dart}{\begin{sideways}Dart\end{sideways}}
\newcommand{\DL}{\begin{sideways}dynamic languages\end{sideways}}
\newcommand{\JSLMS}{\begin{sideways}JS on LMS\end{sideways}}
\newcommand{\SD}{\begin{sideways}ScalaDyno\end{sideways}}

\begin{table}[t!]
\vspace{0.5em}
\begin{tabular}{|l|l|l|l|l|l|l|l|l|l|}
    \hline
                                         & \DS    & \hask  & \dyn   & \SV    & \dart  & \DL    & \SD \\
    \hline
    def. name res. errors                & \xmark & \xmark & \xmark & \xmark & \xmark & \cmark & \cmark \\
%     \hline
%     def. method and field res. errors                 & \cmark & \xmark & \cmark & \xmark & \xmark & \cmark & \cmark \\
    \hline
    def. type errors                     & \cmark & \cmark & \cmark & \cmark & \cmark & \cmark & \cmark \\
    \hline
    fast runtime                         & \xmark & \cmark & \xmark & \cmark & \cmark & \cmark & \cmark \\
    \hline
    compile-time res.                    & \xmark & \cmark & \cmark & \cmark & \xmark & \xmark & \cmark \\
    \hline
\end{tabular}
\caption{Summary of different approaches to deferring type errors. Abbreviations: def. = deferred, res. = resolution}
\vspace{-1em}
\end{table}


\section{Conclusion}

%  \section{Future Work}
%
% The described method for deferring type errors could be extended or combined with other dynamic features such as duck typing. In extension to the existing related work this might be possible by creating {\ttfamily Dynamic} wrapper classes for every class and detyping all instances, fields and method arguments to {\ttfamily Dynamic} such that field resolution could be done at runtime. Of course this would require additional work such as method signature mangling as well as finding an adequate solution for implicit conversions.
%
% One additional feature which is not included in ScalaDyno is the possibility to be less strict about class inheritance such that abstract methods would not need to be implemented by sub-classes unless they were needed at runtime. This would need to be handled in during the {\ttfamily refcheck} phase of the Scala compiler.

 \section{Acknowledgments}
%
 At this point I wanted to thank Vlad Ureche, my supervisor, for his great help with this semester project. By taking time off in his busy schedule, we were able to have weekly meetings where he gave me very valuable feedback without which this project would not have been successful. I also want to thank Damien Engels for his help with hacking around to get the plugin working with sbt. I also thank my friends who read the initial draft of the paper and gave me useful feedback.

\bibliographystyle{abbrvnat}
\bibliography{main} 

\end{document}
