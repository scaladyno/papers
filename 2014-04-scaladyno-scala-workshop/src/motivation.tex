 \section{Approach}

\textbf{This is more like explaining how things happened, rather than describing the motivation.}

 Effort has been invested in including dynamic behaviour into statically typed languages such as, for example, in Haskell and Scala. However, most of the approaches require big changes in the compiler and also runtime supporting libraries which slos down both compilation and execution due to the induced overhead. Our idea was to create a very lightweight tool which might not be as extensible and not cover as many cases as other alternatives but which can still be very fast, easy to implement and easy to use. Our choice for the Scala language was based on the fact that, even though this object-oriented, functional programming language has grown into a full real-world software development language, there is still missing an easy solution for fast development and testing. During the project we also wanted to explore other possible solutions such that the core compiler needed to be modified as least as possible. In the beginning we hoped that we could use the recently added trait {\ttfamily Dynamic} as well as in combination with implicit conversion and typed macros. Sadly we quickly needed to notify that implicit conversion don't apply for conversions to Dynamic. This means that implicit conversion won't automatically apply a conversion to a dynamic type whenever no statically correctly typed alternatives are available. Hence we needed to consider a more integrated solution such that we finally implemented a Scala compiler plugin which does the transformation directly on the AST.
