\section{Theory}

%In the Scala language the the type {\ttfamily Nothing} is a subtype of all other classes and can thus substitute the return value of any type in a method return value for example.
%Similarly the subtyping relations for the error type are the same as for type {\ttfamily Nothing}, which means that they are equivalent except for some bit-flags. Hence an instance of error type, just as one of type {\ttfamily Nothing}, can replace any other value, be it a field, return value or argument without changing or breaking the typing rules.
%Moreover do many typing errors depend on earlier errors such as an object which could not be instantiated and its subsequent field and method accesses. Those errors should not be considered as different ones as they relate to another error. There are many advanced techniques nowadays to eliminate those "spurious" error messages. As cited in Ramsey's [reference] paper, error messages related to an erroneous identifier should be traceable by having their corresponding identifier in the symbol table with type error and by suppressing all the following errors which relate to it. This method helps to catch the spurious errors and to issue only the most relevant error message which identifies the root cause of a specific error.

In order to remove erroneous parts of the AST we have to make some critical assumptions about the types of errors that we can handle. First of all one can note that type errors in Scala, including their side-effects, are localized; they are bound inside a scope defined by a block which might be a method or class body, but the code outside this scope can still be considered correct. However these localized errors can still trigger other (localized) errors in other parts of the code that use the erroneous identifiers, such as instantiations and method calls. In a normal compilation, this should trigger an avalanche of errors in the program.

Yet, in modern compilers, this is not the case: only single, relevant errors are reported to programmers, but the avalanches resulting from these errors are not shown. This has been studied in \cite{spurious} and is currently being applied in most compilers. In Scala, this is done using the |ErrorType|, an alternative bottom type which records the fact that it was produced by an error. Using this marker type combined with specialized typing rules allows the scala compiler to avoid avalanche errors, thus only reporting the relevant error messages.

In the current compiler however a single compile-time error still halts the entire compilation process. This is because the later phases in the compiler rely on the assumption that the tree is correct. Our approach eliminates the erroneous nodes in the tree and the erroneous symbols created while type-checking the program. This allows later phases to compile the program to bytecode which can be correctly executed.

Yet, the cleaned up code may be missing core parts of its functionality, and thus may produce undesired side effects:

\begin{lstlisting-nobreak}
 var path = "/"
 path = s"/home/$user/.config/myfiles"
 //                  ^ value user undefined
 sys.run(s"rm -rf $path") // oh noes...
\end{lstlisting-nobreak}

The example shows that removing parts of the tree is not enough to guarantee safe execution. What we want is to prevent execution past an erroneous node in the tree. This is why Scaladyno replaces erroneous code by throwing runtime exceptions which prevent further code from being executed. These exceptions contain the exact compiler error message that caused the erroneous tree in the first place. After this cleanup, the following compiler phases are able to correctly transform the AST and compile it down to executable bytecode.

Even if some of the type errors have cascading behavior there are still paths through the code that can execute successfully. Allowing these paths to be executed and pass tests is crucial to fast prototyping and refactoring.
