\section{Theory}

%In the Scala language the the type {\ttfamily Nothing} is a subtype of all other classes and can thus substitute the return value of any type in a method return value for example.
%Similarly the subtyping relations for the error type are the same as for type {\ttfamily Nothing}, which means that they are equivalent except for some bit-flags. Hence an instance of error type, just as one of type {\ttfamily Nothing}, can replace any other value, be it a field, return value or argument without changing or breaking the typing rules.
%Moreover do many typing errors depend on earlier errors such as an object which could not be instantiated and its subsequent field and method accesses. Those errors should not be considered as different ones as they relate to another error. There are many advanced techniques nowadays to eliminate those "spurious" error messages. As cited in Ramsey's [reference] paper, error messages related to an erroneous identifier should be traceable by having their corresponding identifier in the symbol table with type error and by suppressing all the following errors which relate to it. This method helps to catch the spurious errors and to issue only the most relevant error message which identifies the root cause of a specific error.

In order to remove erroneous parts of the AST we have to make some main assumptions which define the theory behind those changes.
First of all one can note that type errors in Scala, including their side-effects, are localized; they are bound inside a scope defined by a block which might be a method or class body. However those localized errors can still trigger other (localized) errors in other parts of the code due to instantiations and method calls.

In the current compiler however one single compile-time error still halts the entire compilation process so there should be an easy way to get rid of this behaviour by replacing erroneous nodes by something else. Indeed the method gen.mkSysErrorCall(arg:String) allows to create runtime exceptions of type |Nothing|, which can hence be used to replace entire subtrees of the AST which are erroneous. In the Scala language, type |Nothing| represents the bottom type of type system which means that is it the subtype of any other class and can hence be used as return argument for any method without breaking the subtyping relations.

After this cleanup of the AST, which removed all erroneously typed nodes from the tree, following phases of the compiler are thus able to perform additional transformations on the AST which does not include any erroneous branches any more.

Even if some of the type errors have cascading behaviour there might still be parts in the code which could be executed correctly. We want to keep those parts for the code generating process which is done by pruning the nodes of the AST which are of ErrorType and replace them by a corresponding system error of type Nothing which contain the error message that is usually given at compile-time.