\section{Theory}

In the Scala language the type {\ttfamily Nothing} is a subtype of all other classes and can thus substitute the return value of any type in a method return value for example.
Similarly the error type is the same as the type {\ttfamily Nothing} with some bit-flags set hence it can basically replace any value, be it a field, return value or argument. This property is needed for our solution not to break the static type checker.
Moreover do many typing errors depend on earlier errors such as an object which couldn't be instantiated and its subsequent field and method accesses. Those errors should not be considered as different ones as they relate to the first error. There are many advanced techniques nowadays to eliminate those "spurious" error messages. As cited in Ramsey's [reference] paper, error messages related to an erroneous identifier should be identified by having the corresponding identifier in the symbol type and by suppressing all the following errors. This method helps to catch the errors, to issue only the most relevant error message and identify the root cause of a specific error.