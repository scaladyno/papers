\section{Abstract}

In the academic and the professional community it is agreed on the fact that
both statically and dynamically typed languages have both benefits and
drawbacks. Statically typed languages provide important feedback at compile time
about parts of the program which will go wrong at runtime, however this type
system is also quite strict and thus very unflexible to the developer.
Dynamically typed languages on the other hand leave give a lot more freedom to
they freedom due to the very loose type system where most checks are done at
runtime and the class structures are extensible. It would thus sometimes be
useful to have a programming language with have a blend of the two kinds of
languages with most the benefits combined. Such a language would allow fast
prototyping and could as well provide static guarantees.
ScalaDyno is a take on such a language where the current Scala compiler is
extended such that it deferres naming and typing errors to runtime.